\section{Introduction}
\subsection{Contexte général }
\subsubsection{Client}

Le client, une entreprise active dans le secteur du bâtimentdes travaux publics, situé à Sint-Pieters-Leeuw doit constamment gérer, dans le cadre de ses activités, le flux de matériaux, ses clients, ses employés, ses fournisseurs ainsi que toutes les facturations qui en découlent.

Il ne possède actuellement aucun moyen informatisé lui permettant de gérer l'ensemble de ses entités. Il m'a été demandé de de concevoir une solution pour répondre à l'ensemble de ses besoins :

\begin{itemize}
  \item Gestion clients 
  \item Gestion des Projets
  \item Gestion Matériel
  \item Gestion Stock
  \item Gestion Main d'Oeuvre
  \item Gestion du personnel
  \item Gestion utilisateurs
  \item Gestion Factures
  \item Gestion Devis

\end{itemize}

\subsection{Objectifs }

Il a donc été décidé, en accord avec le client, que la solution sera déployée de telle sorte qu'elle puisse être adaptée au fil des années en fonction de l'évolution des besoins du client. 
De plus, le client étant souvent sur chantier, il est primordial que l'application puisse être facilement portable d'un système / ordinateur à un autre. Il a donc été décidé en commun accord avec le client que la solution sera déployée sous forme d'une application web ce qui apportera une grande flexibilité au niveau de l'utilisation de la solution.

Le volume d'information à gérer étant conséquent, la visualisation doit principalement être adaptée \textbf{aux écrans d'ordinateur.}

\subsection{Cadre didactique de la réalisation}

Je pense que ce projet rentre tout à fait dans le cadre d'un TFE étant donné qu'il require l'analyse, le développement, le déploiement et la maintenance d'une application web afin d'apporter une solution adéquate répondant aux spécificités d'un client.
Le développement se fera à l'aide de technologies modernes qui me permettront d'offrir une interface au goût du jour et modulable. Afin d'améliorer la productivité, je ferais usage de multiples techniques DevOps telles que le déploiement continu et bien d'autres.


